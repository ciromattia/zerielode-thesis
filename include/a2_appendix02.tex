\chapter{Tracciati}
\rhead[\fancyplain{}{\bfseries \thechapter \:Tracciati}]
{\fancyplain{}{\bfseries\thepage}}

In questa appendice sono riportati i tracciati catalografici originali forniti dalla Fondazione Zeri per la compilazione della Scheda F.

\begin{center}
\tiny

\footnotetext[1]{Campo obbligatorio} \footnotetext[2]{Vocabolario: (c) chiuso / (a) aperto}

\begin{longtable}{ | p{1cm} | p{4cm} | p{.6cm} | p{.6cm} | p{5cm} | }
\caption{Tracciato Scheda F} \label{tab:fzeri-schedaF} \\
\hline \multicolumn{1}{|p{1cm}|}{\cellcolor{lightyellow}\textbf{Codice}} & \multicolumn{1}{p{4cm}|}{\cellcolor{lightyellow}\textbf{Significato}} & \multicolumn{1}{p{.6cm}|}{\cellcolor{lightyellow}\textbf{Obb.\footnotemark[1]}} & \multicolumn{1}{p{.6cm}|}{\cellcolor{lightyellow}\textbf{Voc.\footnotemark[2]}} & \multicolumn{1}{p{5cm}|}{\cellcolor{lightyellow}\textbf{Esempio}} \\ \hline
\endfirsthead
\multicolumn{5}{c}%
{{\bfseries \tablename\ \thetable{} -- continua da pag. precedente}} \\
\hline \multicolumn{1}{|p{1cm}|}{\cellcolor{lightyellow}\textbf{Codice}} & \multicolumn{1}{p{4cm}|}{\cellcolor{lightyellow}\textbf{Significato}} & \multicolumn{1}{p{.6cm}|}{\cellcolor{lightyellow}\textbf{Obb.}} & \multicolumn{1}{p{.6cm}|}{\cellcolor{lightyellow}\textbf{Voc.}} & \multicolumn{1}{p{5cm}|}{\cellcolor{lightyellow}\textbf{Esempio}} \\ \hline
\endhead
\hline \multicolumn{5}{r}{{continua a pag. successiva}}\\
\endfoot
\hline \hline
\endlastfoot
  & Didascalia foto &  &  & Detroit Institute of Arts - Angel Gabriel (in ``Annunciation''). Lorenzo di Niccolò Gerini \\ \hline
  OA-F & Collegamento OA-F & x &  & 2055 \\ \hline
  \multicolumn{5}{|l|}{\cellcolor{lightcyan}CODICI} \\ \hline
  TSK & Tipo scheda & x &  & F \\ \hline
  LIR & Livello ricerca & x & x (c) & I \\ \hline
  NCTR & Codice regione & x & x (c) & 08 \\ \hline
  NCTN & Numero catalogo generale & x &  & 16152 \\ \hline
  ESC & Ente schedatore &  &  & Fondazione Federico Zeri – Università di Bologna \\ \hline
  ECP & Ente competente &  &  &  \\ \hline
  \multicolumn{5}{|l|}{\cellcolor{lightcyan}RELAZIONI}\\ \hline
  RVEL & Livello &  &  &  \\ \hline
  RVER & Codice bene radice &  &  &  \\ \hline
  RVES & Codice bene componente &  &  &  \\ \hline
  OGTI & Oggetto di insieme &  &  &  \\ \hline
  \multicolumn{5}{|l|}{\cellcolor{lightcyan}LOCALIZZAZIONE GEOGRAFICO-AMMINISTRATIVA}\\ \hline
  PVCR & Regione & x &  & Emilia-Romagna \\ \hline
  PVCP & Provincia & x &  & BO \\ \hline
  PVCC & Comune & x &  & Bologna \\ \hline
  LDCN & Denominazione del contenitore & x &  & Ex convento di S. Cristina \\ \hline
  LDCU & Denominazione spazio viabilistico & x &  & piazzetta G. Morandi, 2 \\ \hline
  LDCM & Denominazione della raccolta & x &  & Fototeca Zeri \\ \hline
  LDCS & Specifiche &  &  &  \\ \hline
  \multicolumn{5}{|l|}{\cellcolor{lightcyan}UBICAZIONE}\\ \hline
  UBFP & Fondo & x &  & Fototeca Zeri \\ \hline
  UBFS & Serie archivistica & x &  & Pittura italiana \\ \hline
  UBFN & Numero busta & x &  & 0003 \\ \hline
  UBFT & Intestazione busta & x &  & Pittura italiana sec. XIV. Firenze. Giovanni del Biondo, dipinti maggiori \\ \hline
  UBFF & Numero fascicolo & x &  & 1 \\ \hline
  UBFU & Intestazione fascicolo & x &  & Giovanni del Biondo: dipinti maggiori 1 \\ \hline
  UBFC & Collocazione & x &  & PI\_0051/1/7 \\ \hline
  INVN & Numero inventario generale & x &  & 16152 \\ \hline
  INVD & Data inventariazione &  &  &  \\ \hline
  \multicolumn{5}{|l|}{\cellcolor{lightcyan}ALTRE LOCALIZZAZIONI (paragrafo ripetitivo)}\\ \hline
  TCL & Tipo di localizzazione &  & x (c) & provenienza \\ \hline
  PRVS & Stato &  & x (a) & Italia \\ \hline
  PRVP & Provincia &  & x (a) & FI \\ \hline
  PRVC & Comune &  & x (a) & Firenze \\ \hline
  PRL & Altra località/ Località estera &  &  &  \\ \hline
  PRCD & Denominazione &  & x (a) &  \\ \hline
  PRCM & Denominazione della raccolta &  & x (a) & Collezione privata Sandberg Vavalà Evelyn \\ \hline
  PRCI & Numero di inventario &  &  &  \\ \hline
  PRDI & Data ingresso &  &  &  \\ \hline
  PRDU & Data uscita &  &  &  \\ \hline
  \multicolumn{5}{|l|}{\cellcolor{lightcyan}OGGETTO} \\ \hline
  OGTD & Definizione dell’oggetto & x & x (c) & positivo \\ \hline
  OGTB & Natura biblioteconomica dell’oggetto & x & x (c) & m \\ \hline
  OGTS & Forma specifica dell’oggetto &  & x (a) &  \\ \hline
  QNTN & Numero oggetti/elementi & x &  & 1 \\ \hline
  \multicolumn{5}{|l|}{\cellcolor{lightcyan}SOGGETTO (paragrafo ripetitivo)}\\ \hline
  SGTI & Identificazione del soggetto & x &  & Angelo annunciante - cuspide di polittico \\ \hline
  \multicolumn{5}{|l|}{\cellcolor{lightcyan}TITOLO (paragrafo ripetitivo)}\\ \hline
  SGLT & Titolo proprio & (x) &  & Angel Gabriel (in ``Annunciation''). Lorenzo di Niccolò Gerini \\ \hline
  SGLL & Titolo parallelo &  &  &  \\ \hline
  SGLA & Titolo attribuito & (x) &  &  \\ \hline
  SGLS & Specifiche del titolo & x &  & manoscritto sul verso \\ \hline
  \multicolumn{5}{|l|}{\cellcolor{lightcyan}LUOGO E DATA DELLA RIPRESA}\\ \hline
  LRCS & Stato &  & x (a) &  \\ \hline\
  LRCC & Comune &  & x (a) &  \\ \hline
  LRA & Altra località/località estera &  &  &  \\ \hline
  LRO & Occasione &  &  &  \\ \hline
  LRD & Data &  &  & 1930/ ca. \\ \hline
  \multicolumn{5}{|l|}{\cellcolor{lightcyan}CRONOLOGIA} \\ \hline
  DTZG & Secolo & x & x (c) & XX \\ \hline
  DTSI & Da & x &  & 1929 \\ \hline
  DTSV & Validità &  & x (c) & post \\ \hline
  DTSF & A & x &  & 1950 \\ \hline
  DTSL & Validità &  & x (c) & ante \\ \hline
  DTMM & Motivazione & x & x (a) & analisi storica/ analisi tecnico-formale \\ \hline
  DTMS & Specifiche &  &  &  \\ \hline
  \multicolumn{5}{|l|}{\cellcolor{lightcyan}AUTORE DELLA FOTOGRAFIA (paragrafo ripetitivo)} \\ \hline
  AUFN & Autore della fotografia (autore personale) & (x) &  &  \\ \hline
  AUFB & Autore della fotografia (ente collettivo) & (x) &  & Detroit Institute of Arts \\ \hline
  AUFI & Indicazione del nome e dell’indirizzo &  &  & Photographic Dept. Detroit Institute of Arts \\ \hline
  AUFA & Dati anagrafici/estremi cronologici &  &  &  \\ \hline
  AUFS & Riferimento all’autore &  & x (a) &  \\ \hline
  AUFR & Riferimento all’intervento &  & x (a) &  \\ \hline
  AUFM & Motivazione dell’attribuzione & x & x (a) & timbro \\ \hline
  AUFK & Specifiche sull’attribuzione &  &  &  \\ \hline
  \multicolumn{5}{|l|}{\cellcolor{lightcyan}ALTRO AUTORE (paragrafo ripetitivo)} \\ \hline
  AUTN & Nome scelto (autore personale) &  &  & Giovanni del Biondo \\ \hline
  AUTB & Altro autore (ente collettivo) &  &  & Scuola italiana, scuola toscana, scuola fiorentina \\ \hline
  AUTI & Indicazione del nome &  &  &  \\ \hline
  AUTR & Riferimento all’intervento &  & x (a) &  \\ \hline
  \multicolumn{5}{|l|}{\cellcolor{lightcyan}PRODUZIONE E DIFFUSIONE (paragrafo ripetitivo)} \\ \hline
  PDFN & Nome scelto (personale) &  & x (a) &  \\ \hline
  PDFB & Nome scelto (ente collettivo) &  & x (a) &  \\ \hline
  PDFI & Indicazione del nome e dell’indirizzo &  &  &  \\ \hline
  PDFR & Riferimento al ruolo &  & x (a) &  \\ \hline
  PDFL & Luogo &  &  &  \\ \hline
  PDFD & Data &  &  &  \\ \hline
  PDFM & Motivazione dell’attribuzione &  & x (a) &  \\ \hline
  PDFK & Specifiche sull’attribuzione &  &  &  \\ \hline
  EDIT & Denominazione propria &  &  &  \\ \hline
  EDIR & Indicazione di responsabilità (“editor”) &  &  &  \\ \hline
  SFIT & Titolo della serie &  &  &  \\ \hline
  \multicolumn{5}{|l|}{\cellcolor{lightcyan}RAPPORTO} \\ \hline
  ROFF & Stadio opera &  & x (a) &  \\ \hline
  ROFO & Opera iniziale/finale &  &  &  \\ \hline
  ROFC & Collocazione opera iniziale/finale &  &  &  \\ \hline
  ROFI & Inventario opera iniziale/finale &  &  & 1753 \\ \hline
  \multicolumn{5}{|l|}{\cellcolor{lightcyan}DATI TECNICI} \\ \hline
  MTX & Indicazione di colore & x & x (c) & BN \\ \hline
  MTC & Materia e tecnica & x & x (c) & gelatina ai sali d'argento/ carta baritata \\ \hline
  MISO & Tipo misure & x & x (a) & supporto primario \\ \hline
  MISU & Unità di misura & x & x (c) & mm \\ \hline
  MISA & Altezza & x &  & 227 \\ \hline
  MISL & Larghezza & x &  & 189 \\ \hline
  MISS & Spessore &  &  &  \\ \hline
  MISD & Diametro &  &  &  \\ \hline
  \multicolumn{5}{|l|}{\cellcolor{lightcyan}CONSERVAZIONE} \\ \hline
  STCC & Stato di conservazione &  & x (c) & discreto \\ \hline
  STCS & Indicazioni specifiche &  &  & pieghe \\ \hline
  \multicolumn{5}{|l|}{\cellcolor{lightcyan}CONDIZIONE GIURIDICA} \\ \hline
  CDGG & Indicazione generica &  &  & proprietà Ente pubblico non territoriale \\ \hline
  CDGS & Indicazione specifica &  &  & Alma Mater Studiorum Università di Bologna \\ \hline
  \multicolumn{5}{|l|}{\cellcolor{lightcyan}DIRITTI D’AUTORE} \\ \hline
  CPRN & Nome &  &  &  \\ \hline
  CPRD & Data di scadenza &  &  &  \\ \hline
  \multicolumn{5}{|l|}{\cellcolor{lightcyan}FONTI E DOCUMENTI DI RIFERIMENTO (paragrafo ripetitivo)} \\ \hline
  FTAX & Genere & x & x (c) & allegata \\ \hline
  FTAP & Tipo &  &  & fotografia digitale \\ \hline
  FTAD & Data &  &  &  \\ \hline
  FTAN & Nome del file digitale & x &  & \textbackslash40000\textbackslash16400\textbackslash16152.jpg \\ \hline
  FTAT & Note &  &  & insieme \\ \hline
  Verso & Verso della foto &  & x (c) & pubblico \\ \hline
  \multicolumn{5}{|l|}{\cellcolor{lightcyan}ANNOTAZIONI} \\ \hline
  OSS & Osservazioni &  &  &  \\ \hline
  \multicolumn{5}{|c|}{\cellcolor{lightcyan}COMPILAZIONE} \\ \hline
  CMPD & Data & x &  & 37972 \\ \hline
  CMPN & Nome compilatore & x &  & test \\ \hline
  \multicolumn{5}{|l|}{\cellcolor{lightcyan}AGGIORNAMENTO} \\ \hline
  AGGD & Data & x &  & 37972 \\ \hline
  AGGN & Nome Revisore & x &  & test \\ \hline
\end{longtable}

\newpage

\footnotetext[3]{Campo obbligatorio} \footnotetext[4]{Vocabolario: (c) chiuso / (a) aperto}

\begin{longtable}{ | p{1cm} | p{4cm} | p{.6cm} | p{.6cm} | p{5cm} | }
\caption{Tracciato Scheda OA} \label{tab:fzeri-schedaOA} \\
\hline \multicolumn{1}{|p{1cm}|}{\cellcolor{lightyellow}\textbf{Codice}} & \multicolumn{1}{p{4cm}|}{\cellcolor{lightyellow}\textbf{Significato}} & \multicolumn{1}{p{.6cm}|}{\cellcolor{lightyellow}\textbf{Obb.\footnotemark[3]}} & \multicolumn{1}{p{.6cm}|}{\cellcolor{lightyellow}\textbf{Voc.\footnotemark[4]}} & \multicolumn{1}{p{5cm}|}{\cellcolor{lightyellow}\textbf{Esempio}} \\ \hline
\endfirsthead
\multicolumn{5}{c}%
{{\bfseries \tablename\ \thetable{} -- continua da pag. precedente}} \\
\hline \multicolumn{1}{|p{1cm}|}{\cellcolor{lightyellow}\textbf{Codice}} & \multicolumn{1}{p{4cm}|}{\cellcolor{lightyellow}\textbf{Significato}} & \multicolumn{1}{p{.6cm}|}{\cellcolor{lightyellow}\textbf{Obb.}} & \multicolumn{1}{p{.6cm}|}{\cellcolor{lightyellow}\textbf{Voc.}} & \multicolumn{1}{p{5cm}|}{\cellcolor{lightyellow}\textbf{Esempio}} \\ \hline
\endhead
\hline \multicolumn{5}{r}{{continua a pag. successiva}}\\
\endfoot
\hline \hline
\endlastfoot
  IDC & Identificazione convenzionale & x &  & Giovanni del Biondo - sec. XIV - Angelo annunciante - 2055 \\ \hline
  \multicolumn{5}{|l|}{\cellcolor{lightcyan}CODICI}  \\ \hline
  N. scheda & Numero scheda & x &  & 2055 \\ \hline
  RVEL & Livello &  &  &   \\ \hline
  \multicolumn{5}{|l|}{\cellcolor{lightcyan}UBICAZIONE FOTO DI RIFERIMENTO (paragrafo ripetitivo)}  \\ \hline
  UBFS & Serie  & x & x (a) & Pittura italiana \\ \hline
  UBFN & Numero busta & x & x (a) & 0051 \\ \hline
  UBFF & Numero fascicolo & x & x (a) & 1 \\ \hline
  \multicolumn{5}{|l|}{\cellcolor{lightcyan}AUTORE (paragrafo ripetitivo)}  \\ \hline
  AUTN & Nome & x &  & Giovanni del Biondo \\ \hline
  AUTS & Riferimento all’autore &  & x (c) &  \\ \hline
  ATBD & Ambito culturale & x & x (a) & Scuola italiana, scuola toscana, scuola fiorentina \\ \hline
  AUTM & Motivazione dell’attribuzione & x & x (a) & Classificazione F. Zeri//Bibliografia \\ \hline
  \multicolumn{5}{|l|}{\cellcolor{lightcyan}ALTRE ATTRIBUZIONI (paragrafo ripetitivo)}  \\ \hline
  AAT & Nome altro autore &  &  & Lorenzo di Niccolò  \\ \hline
  AATS & Riferimento all’altro autore &  & x (c) &   \\ \hline
  AATM & Motivazione dell’attribuzione &  & x (a) & Nota anonima sul verso della fotografia  \\ \hline
  \multicolumn{5}{|l|}{\cellcolor{lightcyan}RIFERIMENTO AD ALTRE SCHEDE}  \\ \hline
  RSEC & Oggetto di insieme &  &  & Polittico smembrato di Giovanni del Biondo per l'Oratorio di S. Maria delle Grazie a San Giovanni Valdarno  \\ \hline
  \multicolumn{5}{|l|}{\cellcolor{lightcyan}OGGETTO}  \\ \hline
  OGTD & Definizione & x & x (a) & cuspide di polittico  \\ \hline
  OGTV & Identificazione &  & x (a) & elemento d'insieme  \\ \hline
  OGTT & Tipologia & x & x (c) & dipinto  \\ \hline
  \multicolumn{5}{|l|}{\cellcolor{lightcyan}SOGGETTO  (paragrafo ripetitivo)}  \\ \hline
  SGTI & Titolo & x & x (a) & Adorazione dei pastori  \\ \hline
  SGTT & Denominazione/titolo tradizionale &  &  &   \\ \hline
  \multicolumn{5}{|l|}{\cellcolor{lightcyan}MATERIA E TECNICA (paragrafo ripetitivo)}  \\ \hline
  MTC & Materia e tecnica &  & x (a) & Angelo annunciante  \\ \hline
  \multicolumn{5}{|l|}{\cellcolor{lightcyan}MISURE} \\ \hline
  MISU & Unità di misura &  & x (c) & cm  \\ \hline
  MISA & Altezza &  &  & 35.5  \\ \hline
  MISL & Larghezza &  &  & 17.8  \\ \hline
  MISP & Profondità &  &  &   \\ \hline
  MISD & Diametro &  &  &   \\ \hline
  MISV & Varie &  &  &   \\ \hline
  \multicolumn{5}{|l|}{\cellcolor{lightcyan}RAPPORTO OPERA FINALE/ORIGINALE}  \\ \hline
  ROFF & Stadio opera &  & x (a) &   \\ \hline
  ROFO & Opera finale/originale &  &  &   \\ \hline
  ROFS & Soggetto opera finale/originale &  & x (a) &   \\ \hline
  ROFA & Autore opera finale/originale &  &  &   \\ \hline
  ROFD & Datazione opera finale/originale &  &  &   \\ \hline
  ROFC & Collocazione opera finale/originale &  &  &   \\ \hline
  \multicolumn{5}{|l|}{\cellcolor{lightcyan}LOCALIZZAZIONE GEOGRAFICO-AMMINISTRATIVA} \\ \hline
  PVCS & Stato &  & x (a) & Stati Uniti d'America  \\ \hline
  PVCR & Regione/ Stato federale &  & x (a) & Michigan  \\ \hline
  PVCP & Provincia &  & x (c) &   \\ \hline
  PVCC & Comune &  & x (a) & Detroit (MI)  \\ \hline
  PVCL & Località &  & x (a) &   \\ \hline
  LDCN & Denominazione del contenitore &  & x (a) & Detroit Institute of Arts  \\ \hline
  LDCS & Localizzazione specifica &  &  & Acc. No. 29.315  \\ \hline
  \multicolumn{5}{|l|}{\cellcolor{lightcyan}ALTRE LOCALIZZAZIONI (paragrafo ripetitivo)}  \\ \hline
  PRVS & Stato &  & x (a) & Italia  \\ \hline
  PRVR & Regione &  & x (a) & Toscana  \\ \hline
  PRVP & Provincia &  & x (c) & FI  \\ \hline
  PRVC & Comune &  & x (a) & Firenze  \\ \hline
  PRVL & Località &  & x (a) &   \\ \hline
  PRCD & Denominazione del contenitore &  & x (a) & L. Grassi  \\ \hline
  PRCS & Localizzazione specifica &  &  &   \\ \hline
  PRDI & Data ingresso &  &  &   \\ \hline
  PRDU & Data uscita &  &  & 1929  \\ \hline
  \multicolumn{5}{|l|}{\cellcolor{lightcyan}CRONOLOGIA (paragrafo ripetitivo)} \\ \hline
  DTZG & Indicazione generica & x &  & sec. XIV  \\ \hline
  DTZS & Frazione di secolo &  & x (c) & terzo quarto  \\ \hline
  DTSI & Da & x &  & 1365  \\ \hline
  DTSV & Validità &  & x (c) &   \\ \hline
  DTSF & A & x &  & 1370  \\ \hline
  DTSL & Validità &  & x (c) &   \\ \hline
  \multicolumn{5}{|l|}{\cellcolor{lightcyan}ALTRE DATAZIONI (pragrafo ripetitivo)} \\ \hline
  ADT & Altre datazioni &  &  & sec. XV (1400-1449)  \\ \hline
  \multicolumn{5}{|l|}{\cellcolor{lightcyan}ICONCLASS} \\ \hline
  DESI & Codice Iconclass &  &  &   \\ \hline
  DESS & Indicazioni sul soggetto &  &  & Angelo  \\ \hline
  \multicolumn{5}{|l|}{\cellcolor{lightcyan}BIBLIOGRAFIA (paragrafo ripetitivo)} \\ \hline
  BIBX & Genere &  & x (c) & bibliografia specifica  \\ \hline
  BIBA & Autore &  &  & Berenson B.  \\ \hline
  BIBG & Libro/ Rivista &  &  & Italian Pictures of the Renaissance  \\ \hline
  BIBT & Titolo contributo &  &  &   \\ \hline
  BIBD & Anno di edizione &  &  & 1932  \\ \hline
  BIBN & Pagine specifiche &  &  & p. 240  \\ \hline
  \multicolumn{5}{|l|}{\cellcolor{lightcyan}ALLEGATI (paragrafo ripetitivo)} \\ \hline
  FNTI & Codice identificativo &  &  & F134  \\ \hline
  FNTP & Tipo &  &  & lettera  \\ \hline
  FNTA & Autore &  &  & Contini Bonacossi A.  \\ \hline
  FNTT & Denominazione &  &  & Lettera dattiloscritta di Alessandro Contini Bonacossi a Federico Zeri contenente considerazioni sui due ``Profeti'' di Giovanni del Biondo già parte del polittico dell'Oratorio di S. Maria delle Grazie a San Giovanni Valdarno, transitati dalla Collezione Kress e ora al Museo de Arte de Ponce.  \\ \hline
  FNTD & Data &  &  & 1963/08/02  \\ \hline
  FNTS & Segnatura &  &  & PI\_0051/1/1-13  \\ \hline
  \multicolumn{5}{|l|}{\cellcolor{lightcyan}MOSTRE (Paragrafo ripetitivo)} \\ \hline
  MSTT & Titolo &  &  &   \\ \hline
  MSTL & Luogo &  &  &   \\ \hline
  MSTD & Data &  &  &   \\ \hline
  \multicolumn{5}{|l|}{\cellcolor{lightcyan}OSSERVAZIONI} \\ \hline
  OSS & Osservazioni &  &  & Foto INVN 16152, verso: note anonime a matita in alto: ``Angel Gabriel / (in ``Annunciation'') / Lorenzo di Niccolo Gerini''; in alto a destra: ``Vavalà''  \\ \hline
  NOTE &  &  &  &   \\ \hline
  NOTE & Note &  &  &   \\ \hline
  \multicolumn{5}{|l|}{\cellcolor{lightcyan}FOTO ALLEGATE (paragrafo ripetitivo)} \\ \hline
  FTAP & Tipo &  & x (c) & fotografia digitale  \\ \hline
  FTAD & Data &  &  &   \\ \hline
  FTAN & Nome file digilate &  &  & \textbackslash40000\textbackslash16400\textbackslash16152.jpg  \\ \hline
  FTAT & Note &  &  & insieme  \\ \hline
  Verso & Verso della foto &  & x (c) & Pubblico  \\ \hline
\end{longtable}

\footnotetext[5]{Campo ripetibile} \footnotetext[6]{Campo obbligatorio} \footnotetext[7]{Vocabolario: (c) chiuso / (a) aperto}

\begin{longtable}{ | p{1cm} | p{4cm} | p{.6cm} | p{.6cm} | p{.6cm} | p{5cm} | }
\caption{Tracciato archivio: descrizione archivio} \label{tab:fzeri-archivio-desc} \\
\hline \multicolumn{1}{|p{1cm}|}{\cellcolor{lightyellow}\textbf{Codice}} & \multicolumn{1}{p{4cm}|}{\cellcolor{lightyellow}\textbf{Significato}} & \multicolumn{1}{p{.6cm}|}{\cellcolor{lightyellow}\textbf{Rep.\footnotemark[5]}} & \multicolumn{1}{p{.6cm}|}{\cellcolor{lightyellow}\textbf{Obb.\footnotemark[6]}} & \multicolumn{1}{p{.6cm}|}{\cellcolor{lightyellow}\textbf{Voc.\footnotemark[7]}} & \multicolumn{1}{p{5cm}|}{\cellcolor{lightyellow}\textbf{Esempio}} \\ \hline
\endfirsthead
\multicolumn{6}{c}%
{{\bfseries \tablename\ \thetable{} -- continua da pag. precedente}} \\
\hline \multicolumn{1}{|p{1cm}|}{\cellcolor{lightyellow}\textbf{Codice}} & \multicolumn{1}{p{4cm}|}{\cellcolor{lightyellow}\textbf{Significato}} & \multicolumn{1}{p{.6cm}|}{\cellcolor{lightyellow}\textbf{Rep.\footnotemark[5]}} & \multicolumn{1}{p{.6cm}|}{\cellcolor{lightyellow}\textbf{Obb.\footnotemark[6]}} & \multicolumn{1}{p{.6cm}|}{\cellcolor{lightyellow}\textbf{Voc.\footnotemark[7]}} & \multicolumn{1}{p{5cm}|}{\cellcolor{lightyellow}\textbf{Esempio}} \\ \hline
\endhead
\hline \multicolumn{6}{r}{{continua a pag. successiva}}\\
\endfoot
\hline \hline
\endlastfoot
   Archivio & Archivio &  & x &  & Fondazione Federico Zeri - Università di Bologna \\ \hline
   UBFP & Fondo &  & x & x (a) & Fototeca Zeri \\ \hline
   UBFS & Serie &  & x & x (a) & Pittura italiana \\ \hline
   UBFN & Numero busta &  & x & x (a) & 0051 \\ \hline
   UBFT & Intestazione busta &  & x & x (a) & Pittura italiana sec. XIV. Firenze. Giovanni del Biondo, dipinti maggiori \\ \hline
   UBFF & Numero fascicolo &  & x & x (a) & 1 \\ \hline
   UBFU & Intestazione fascicolo &  & x & x (a) & Giovanni del Biondo: dipinti maggiori 1 \\ \hline
   Consistenza & Consistenza del fascicolo (foto) &  & x &  & 58 \\ \hline
   Consistenza & Consistenza del fascicolo (allegati) &  &  &  & 9 \\ \hline
\end{longtable}

\begin{longtable}{ | p{1cm} | p{4cm} | p{.6cm} | p{.6cm} | p{.6cm} | p{5cm} | }
\caption{Tracciato archivio: autori opere} \label{tab:fzeri-archivio-aut} \\
\hline \multicolumn{1}{|p{1cm}|}{\cellcolor{lightyellow}\textbf{Codice}} & \multicolumn{1}{p{4cm}|}{\cellcolor{lightyellow}\textbf{Significato}} & \multicolumn{1}{p{.6cm}|}{\cellcolor{lightyellow}\textbf{Rep.\footnotemark[5]}} & \multicolumn{1}{p{.6cm}|}{\cellcolor{lightyellow}\textbf{Obb.\footnotemark[6]}} & \multicolumn{1}{p{.6cm}|}{\cellcolor{lightyellow}\textbf{Voc.\footnotemark[7]}} & \multicolumn{1}{p{5cm}|}{\cellcolor{lightyellow}\textbf{Esempio}} \\ \hline
\endfirsthead
\multicolumn{6}{c}%
{{\bfseries \tablename\ \thetable{} -- continua da pag. precedente}} \\
\hline \multicolumn{1}{|p{1cm}|}{\cellcolor{lightyellow}\textbf{Codice}} & \multicolumn{1}{p{4cm}|}{\cellcolor{lightyellow}\textbf{Significato}} & \multicolumn{1}{p{.6cm}|}{\cellcolor{lightyellow}\textbf{Rep.\footnotemark[5]}} & \multicolumn{1}{p{.6cm}|}{\cellcolor{lightyellow}\textbf{Obb.\footnotemark[6]}} & \multicolumn{1}{p{.6cm}|}{\cellcolor{lightyellow}\textbf{Voc.\footnotemark[7]}} & \multicolumn{1}{p{5cm}|}{\cellcolor{lightyellow}\textbf{Esempio}} \\ \hline
\endhead
\hline \multicolumn{6}{r}{{continua a pag. successiva}}\\
\endfoot
\hline \hline
\endlastfoot
   AUTH & Sigla per citazione &  & x &  & 10005477 \\ \hline
   AUTQ & Qualifica &  &  &  & pittore \\ \hline
   AUTN & Autore &  & x &  & Giovanni del Biondo \\ \hline
   AUTA & Dati anagrafici &  & x &  & notizie dal 1356/ 1398 \\ \hline
   AUTC & Cognome &  &  &  &  \\ \hline
   AUTO & Nome &  &  &  & Giovanni \\ \hline
   AUTP & Pseudonimo &  &  &  &  \\ \hline
   AUTE & Nome convenzionale &  &  &  &  \\ \hline
   AUTZ & Sesso &  &  & x (c) &  \\ \hline
   AUTV & Varianti &  &  &  & Giovanni dei Landini (?)/ Maestro della Cappella Rinuccini \\ \hline
   AUTG & Luogo e/o periodo di attività &  &  &  &  \\ \hline
   AUTU & Scuola di appartenenza &  &  &  & scuola fiorentina \\ \hline
   AUTL & Luogo di nascita &  &  &  & Pratovecchio \\ \hline
   AUTD & Data di nascita &  &  &  &  \\ \hline
   AUTX & Luogo di morte &  &  &  & Firenze \\ \hline
   AUTT & Data di morte &  &  &  & 1398 \\ \hline
\end{longtable}

\begin{longtable}{ | p{1cm} | p{4cm} | p{.6cm} | p{.6cm} | p{.6cm} | p{5cm} | }
\caption{Tracciato archivio: bibliografia} \label{tab:fzeri-archivio-biblio} \\
\hline \multicolumn{1}{|p{1cm}|}{\cellcolor{lightyellow}\textbf{Codice}} & \multicolumn{1}{p{4cm}|}{\cellcolor{lightyellow}\textbf{Significato}} & \multicolumn{1}{p{.6cm}|}{\cellcolor{lightyellow}\textbf{Rep.\footnotemark[5]}} & \multicolumn{1}{p{.6cm}|}{\cellcolor{lightyellow}\textbf{Obb.\footnotemark[6]}} & \multicolumn{1}{p{.6cm}|}{\cellcolor{lightyellow}\textbf{Voc.\footnotemark[7]}} & \multicolumn{1}{p{5cm}|}{\cellcolor{lightyellow}\textbf{Esempio}} \\ \hline
\endfirsthead
\multicolumn{6}{c}%
{{\bfseries \tablename\ \thetable{} -- continua da pag. precedente}} \\
\hline \multicolumn{1}{|p{1cm}|}{\cellcolor{lightyellow}\textbf{Codice}} & \multicolumn{1}{p{4cm}|}{\cellcolor{lightyellow}\textbf{Significato}} & \multicolumn{1}{p{.6cm}|}{\cellcolor{lightyellow}\textbf{Rep.\footnotemark[5]}} & \multicolumn{1}{p{.6cm}|}{\cellcolor{lightyellow}\textbf{Obb.\footnotemark[6]}} & \multicolumn{1}{p{.6cm}|}{\cellcolor{lightyellow}\textbf{Voc.\footnotemark[7]}} & \multicolumn{1}{p{5cm}|}{\cellcolor{lightyellow}\textbf{Esempio}} \\ \hline
\endhead
\hline \multicolumn{6}{r}{{continua a pag. successiva}}\\
\endfoot
\hline \hline
\endlastfoot
   BIBH & Sigla per citazione &  & x &  &  \\ \hline
   BIBA & Autore &  & x &  & Berenson B. \\ \hline
   BIBC & Curatore &  &  &  &  \\ \hline
   BIBF & Tipo &  &  &  & monografia \\ \hline
   BIBG & Libro/rivista &  & x &  & Italian Pictures of the Renaissance \\ \hline
   BIBT & Titolo contributo &  &  &  &  \\ \hline
   BIBL & Luogo di edizione &  &  &  & Oxford \\ \hline
   BIBZ & Editore &  &  &  & Clarendon Press \\ \hline
   BIBD & Anno di edizione &  &  &  & 1932 \\ \hline
   BIBE & Numero di edizione &  &  &  &  \\ \hline
   BIBS & Specifiche &  &  &  &  \\ \hline
   BIBV & Volume &  &  &  &  \\ \hline
   BIBP & Pagine &  &  &  &  \\ \hline
\end{longtable}

\begin{longtable}{ | p{1cm} | p{4cm} | p{.6cm} | p{.6cm} | p{.6cm} | p{5cm} | }
\caption{Tracciato archivio: allegati} \label{tab:fzeri-archivio-attach} \\
\hline \multicolumn{1}{|p{1cm}|}{\cellcolor{lightyellow}\textbf{Codice}} & \multicolumn{1}{p{4cm}|}{\cellcolor{lightyellow}\textbf{Significato}} & \multicolumn{1}{p{.6cm}|}{\cellcolor{lightyellow}\textbf{Rep.\footnotemark[5]}} & \multicolumn{1}{p{.6cm}|}{\cellcolor{lightyellow}\textbf{Obb.\footnotemark[6]}} & \multicolumn{1}{p{.6cm}|}{\cellcolor{lightyellow}\textbf{Voc.\footnotemark[7]}} & \multicolumn{1}{p{5cm}|}{\cellcolor{lightyellow}\textbf{Esempio}} \\ \hline
\endfirsthead
\multicolumn{6}{c}%
{{\bfseries \tablename\ \thetable{} -- continua da pag. precedente}} \\
\hline \multicolumn{1}{|p{1cm}|}{\cellcolor{lightyellow}\textbf{Codice}} & \multicolumn{1}{p{4cm}|}{\cellcolor{lightyellow}\textbf{Significato}} & \multicolumn{1}{p{.6cm}|}{\cellcolor{lightyellow}\textbf{Rep.\footnotemark[5]}} & \multicolumn{1}{p{.6cm}|}{\cellcolor{lightyellow}\textbf{Obb.\footnotemark[6]}} & \multicolumn{1}{p{.6cm}|}{\cellcolor{lightyellow}\textbf{Voc.\footnotemark[7]}} & \multicolumn{1}{p{5cm}|}{\cellcolor{lightyellow}\textbf{Esempio}} \\ \hline
\endhead
\hline \multicolumn{6}{r}{{continua a pag. successiva}}\\
\endfoot
\hline \hline
\endlastfoot
   \multicolumn{6}{|l|}{\cellcolor{lightcyan}PROVENIENZA}  \\ \hline
   FNTN & Nome Archivio/ Fondo &  & x &  & Fondazione Federico Zeri - Università di Bologna/ Fototeca Zeri \\ \hline
   UBFP & Fondo &  & x &  & Fototeca  Zeri \\ \hline
   UBFS & Serie &  & x &  & Pittura italiana \\ \hline
   UBFN & Numero busta &  & x &  & 0051 \\ \hline
   UBFT & Intestazione busta &  & x &  & Pittura italiana sec. XIV. Firenze. Giovanni del Biondo, dipinti maggiori \\ \hline
   UBFF & Numero fascicolo &  & x &  & 1 \\ \hline
   UBFU & Intestazione fascicolo &  & x &  & Giovanni del Biondo: dipinti maggiori 1 \\ \hline
   FNTS & Segnatura &  & x &  & PI\_0051/1/1-13 \\ \hline
   N. BID & Numero identificativo (BID) del volume collegato al documento &  &  &  &  \\ \hline
   N. Scheda OA & Numero scheda dell’opera d’arte collegata al documento & Si &  &  & 2054 \\ \hline
   \multicolumn{6}{|l|}{\cellcolor{lightcyan}ATTUALE COLLOCAZIONE}  \\ \hline
   FNTO & Collocazione &  &  &  & Allegati Fototeca 3 \\ \hline
   FNTI & Codice identificativo &  & x &  & F134 \\ \hline
   FNTF & Consistenza &  &  &  & 1 \\ \hline
   \multicolumn{6}{|l|}{\cellcolor{lightcyan}TIPOLOGIA E CONTENUTO}  \\ \hline
   FNTP & Tipo &  &  & x (a) & lettera \\ \hline
   FNTA & Autore &  &  &  & Contini Bonacossi A. \\ \hline
   FNTE & Destinatario &  &  &  & Zeri F. \\ \hline
   FNTD & Data &  &  &  & 1963/08/02 \\ \hline
   FNTT & Descrizione &  & x &  & Lettera dattiloscritta di Alessandro Contini Bonacossi a Federico Zeri contenente considerazioni sui due ``Profeti'' di Giovanni del Biondo già parte del polittico dell'Oratorio di S. Maria delle Grazie a San Giovanni Valdarno, transitati dalla Collezione Kress e ora al Museo de Arte de Ponce. \\ \hline
   FNTR & Trascrizione &  &  &  &  \\ \hline
   \multicolumn{6}{|l|}{\cellcolor{lightcyan}IMMAGINI E VISIBILITÀ}  \\ \hline
   FNTJ & Collegamento al file digitale & Si &  &  & \textbackslash{}Allegati\textbackslash{}Fototeca\textbackslash{}f134\_g.jpg \\ \hline
   Pubblico/Privato & Pubblico/ Privato &  &  &  & privato \\ \hline
\end{longtable}

\footnotetext[5]{Campo ripetibile} \footnotetext[6]{Campo obbligatorio} \footnotetext[7]{Vocabolario: (c) chiuso / (a) aperto}

\begin{longtable}{ | p{1cm} | p{4cm} | p{.6cm} | p{.6cm} | p{.6cm} | p{5cm} | }
\caption{Tracciato archivio: autori fotografi} \label{tab:fzeri-archivio-ph} \\
\hline \multicolumn{1}{|p{1cm}|}{\cellcolor{lightyellow}\textbf{Codice}} & \multicolumn{1}{p{4cm}|}{\cellcolor{lightyellow}\textbf{Significato}} & \multicolumn{1}{p{.6cm}|}{\cellcolor{lightyellow}\textbf{Rep.\footnotemark[5]}} & \multicolumn{1}{p{.6cm}|}{\cellcolor{lightyellow}\textbf{Obb.\footnotemark[6]}} & \multicolumn{1}{p{.6cm}|}{\cellcolor{lightyellow}\textbf{Voc.\footnotemark[7]}} & \multicolumn{1}{p{5cm}|}{\cellcolor{lightyellow}\textbf{Esempio}} \\ \hline
\endfirsthead
\multicolumn{6}{c}%
{{\bfseries \tablename\ \thetable{} -- continua da pag. precedente}} \\
\hline \multicolumn{1}{|p{1cm}|}{\cellcolor{lightyellow}\textbf{Codice}} & \multicolumn{1}{p{4cm}|}{\cellcolor{lightyellow}\textbf{Significato}} & \multicolumn{1}{p{.6cm}|}{\cellcolor{lightyellow}\textbf{Rep.\footnotemark[5]}} & \multicolumn{1}{p{.6cm}|}{\cellcolor{lightyellow}\textbf{Obb.\footnotemark[6]}} & \multicolumn{1}{p{.6cm}|}{\cellcolor{lightyellow}\textbf{Voc.\footnotemark[7]}} & \multicolumn{1}{p{5cm}|}{\cellcolor{lightyellow}\textbf{Esempio}} \\ \hline
\endhead
\hline \multicolumn{6}{r}{{continua a pag. successiva}}\\
\endfoot
\hline \hline
\endlastfoot
   AUFH & Sigla per citazione &  &  &  & 10002303 \\ \hline
   AUFQ & Qualifica &  &  &  &  \\ \hline
   AUFN/ AUFB & Autore della foto (personale/ collettivo) &  & x &  & Detroit Institute of Arts \\ \hline
   AUFC & Cognome &  &  &  &  \\ \hline
   AUFO & Nome &  &  &  &  \\ \hline
   AUFP & Pseudonimo &  &  &  &  \\ \hline
   AUFJ & Nazionalità &  &  &  & Statunitense \\ \hline
   AUFZ & Sesso &  &  &  &  \\ \hline
   AUFV & Varianti &  &  &  &  \\ \hline
   AUFG & Luogo e/o periodo di attività &  &  &  & Detroit (Michigan), 5200 Woodward Avenue (1885 - ) \\ \hline
   AUFW & Luoghi conservazione raccolte &  &  &  &  \\ \hline
   AUFL & Luogo di nascita &  &  &  &  \\ \hline
   AUFD & Data di nascita &  &  &  &  \\ \hline
   AUFX & Luogo di morte &  &  &  &  \\ \hline
   AUFT & Data di morte &  &  &  &  \\ \hline
\end{longtable}
\end{center}
\footnotetext[5]{Campo ripetibile} \footnotetext[6]{Campo obbligatorio} \footnotetext[7]{Vocabolario: (c) chiuso / (a) aperto}

