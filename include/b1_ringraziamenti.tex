\chapter*{Ringraziamenti}

Ho sempre pensato che la pagina dei ringraziamenti fosse quella più letta in qualsiasi dissertazione di laurea e al contempo quella più difficile da scrivere per la sua disponibilità a raccogliere santi e fanti, le colonne dei buoni e dei cattivi, discorsi seri e faceti, celebrazioni dovute e omaggi inaspettati.

Altresì ho sempre pensato che fosse un modo forse un po' più rilevante di altri per riconoscere l'importanza di chi ha viaggiato con me e di chi lo sta ancora facendo.

E dopo tredici anni i compagni di viaggio passati e presenti cominciano ad essere molti, e c'è sempre il rischio che ne scappi qualcuno.

In rigoroso ordine sparso comincio quindi a ringraziare chi ha condiviso i primi anni di università, il glorioso \emph{Lab1} con gli iMac colorati battezzati come gruppi Metal, il gruppo \emph{BES} capitanato da Re Enzo, Mattia che mi veniva a svegliare entrando dalla finestra e Mauro che mi ha sopportato finché ce l'ha fatta.

Sicuramente un grosso grazie va a Fabio e a Silvio e Francesca per aver accolto a braccia aperte la lucida follia di riaprire la partita, calare le tredici carte e chiudere in una sola mano. Senza di voi probabilmente non ce l'avrei fatta.

Andrea e Alberto sono sempre lì, nell'appartamentino che si sono costruiti da qualche parte in me, nonostante i silenzi e i distacchi. E così Gianluca, così diverso da me da rendere il mio legame con lui irrinunciabile.

Non posso non menzionare \emph{RoART} tutta, le notti passate a organizzare \emph{RO-Woodstock}, le riunioni, i chitarristi sulla camionetta dei pompieri, le foto e la musica. E non posso non rivolgere un pensiero particolare a Enrico e Francesca, loro sanno perché.

A ventisei anni ho scoperto il teatro, le sue gioie e i suoi dolori; \emph{Nexus} ormai è una seconda casa ed è grazie a chi ne fa parte, a chi mi affianca in scena e fuori, che la mia voglia di continuare il cammino cresce sempre più: Demis e Barbara, sapete di averne il merito e la colpa.

\emph{Comperio} è stata la mia università quando ho deciso di abbandonare Bologna, grazie a Dario e Paolo che mi hanno insegnato un lavoro; e un doveroso grazie a Marco, Isacco, Giulio e Nicolò per i confronti, i bisticci, i discorsi e per le innumerevoli puttanate che hanno accompagnato le pause caffè.

Grazie a Matteo per tutte le avventure passate insieme, assieme ad Andrea per le serate da Severo, per le LAN e per i compleanni al cinese.

Grazie ai miei santoli Anna e Gianni per essere sempre stati i miei secondi genitori; a nonna Maria per gli ovetti e per averle fatto venire i capelli bianchi, e a nonna Rina, ché son sicuro che semmai qualcosa esiste, lassù, ogni tanto butta un occhio su di me per merito suo.

Grazie a Sarah per gli anni passati insieme, per le cose buone che ci sono state, nonostante tutto.

Grazie a Cecilia per esserci stata sempre, ogni singola volta in cui sono inciampato, e per il coraggio e la schiettezza che avrei sempre voluto mi fossero d'esempio.

A Silvia, per essere cresciuta con me, per avermi accompagnato negli anni dell'università e in quelli a seguire, per aver toccato il cielo con un dito e per mille e mille altre cose. E per esserci di nuovo, finalmente.

A Sara per avermi raccolto, per essere stata la scintilla che ha riacceso il fuoco, ché senza di lei non sareste a leggere queste parole. Per avermi permesso nuovamente di credere che tutto è possibile.

Grazie a Martina per le furiose litigate e i larghi sorrisi, so che non è stato affatto facile avere un fratello come me.

Grazie a Celeste per un disegno che ancora conservo e che ho riguardato commosso a gennaio.

Ho sempre creduto nell'importanza delle ultime righe, e per questo vanno a chi mi ha permesso di essere quello che sono, a chi mi ha regalato risorse ma soprattutto esperienza, a chi si è scontrato e incontrato con me, a chi è cresciuto assieme a me. A chi credo non abbia mai perso veramente la speranza di vedermi con una corona d'alloro in testa: Paola e Luciano.\\
Sono fiero di essere vostro figlio.
